\section{Preliminaries}
	
	\subsection{Kan extensions}
	
	Given categories $\mathsf{B}$ and $\C$ we denote their associated \textit{functor category} by $\Fun(\mathsf{B}, \C)$.
	Recall that a category is said to be \textit{small} if its objects and morphisms form sets.
	We denote the category of small categories by $\Cat$.
	A category is said to be \textit{cocomplete} if any functor to it from a small category has a colimit.
	If $\mathsf{A}$ is small and $\mathsf{C}$ cocomplete, then the \textit{(left) Kan extension of $g$ along $f$} exists for any pair of functors $f$ and $g$ in the diagram below, and it is the initial object in $\Fun(\mathsf{B}, \mathsf{C})$ making
	\begin{equation*}
	\begin{tikzcd}[column sep=normal, row sep=normal]
	\mathsf{A} \arrow[d, "f"'] \arrow[r, "g"] & \mathsf{C} \\ 
	\mathsf{B} \arrow[dashed, ur, bend right] & \quad 
	\end{tikzcd}
	\end{equation*}
	commute.
	
	\subsection{Chain complexes}
	
	Fix a commutative and unital ring $\k$ and denoted by $\Ch$ the category of differential graded $\k$-modules with the homological convention.
	Its objects and morphisms are referred to as \textit{chain complexes} and \textit{chain maps} respectively.
	For chain complexes $C$ and $C^\prime$ we regard, as usual, their product $C \otimes C^\prime$ and set of linear maps $\Hom(C, C^\prime)$ as a chain complexes, making $\Ch$ into a closed symmetric monoidal category.
	The $i\th$ \textit{suspension} functor $s^i \colon \Ch \to \Ch$ is defined at the level of graded modules by $(s^{i}M)_n = M_{n-i}$.
	
	\subsection{Simplicial sets}
	
	For any non-negative integer $n$ denote by $[n]$ the category generated by the poset $\{0 \leq 1 \leq \dots \leq n\}$.
	The \textit{simplex category} $\simplex$ is the category with set of objects $\big\{ [n] \big\}$ morphisms given by functors.
	These are generated by the usual (simplicial) \textit{coface} and \textit{codegeneracy maps}
	\begin{equation*}
	\delta_i \colon [n-1] \to [n], \qquad \sigma_i \colon [n+1] \to [n]
	\end{equation*}
	for $j \in \{0, \dots, n\}$.
	
	The category of \textit{simplicial sets} is the functor category $\sSet = \Fun(\simplex^\op, \Set)$.
	The \textit{standard $n$-simplex} is the simplicial set $\simplex^n = \simplex(-, [n])$, and the \textit{Yoneda embedding} $\Y \colon \simplex \to \sSet$ is the functor induced by $[n] \mapsto \simplex^n$.
	Additionally, for any simplicial set $X$ we write
	\begin{equation*}
	X([n]) = X_n, \qquad
	X(\delta_i) = d_i, \qquad
	X(\sigma_i) = s_i,
	\end{equation*}
	and remark that
	\begin{equation*}
	X_n \cong \colim_{\simplex^n \to X} \simplex^n.
	\end{equation*}
	
	\subsection{Simplicial chains} \label{ss:simplicial sets}
	
	For non-negative integers $m$ and $n$, let $\simplex_{\deg} \big( [m], [n] \big)$ be the subset of \textit{degenerate morphisms} in $\simplex \big( [m], [n] \big)$, i.e., those of the form $\sigma_i \circ \tau$ with $\tau$ any morphism in $\simplex \big( [m], [n+1] \big)$.
	The functor of (simplicial) \textit{chains} $\chains \colon \sSet \to \Ch$ is the Kan extension along the Yoneda embedding of the functor $\simplex \to \Ch$ defined next.
	To an object $[n]$ it assigns the chain complex having in degree $m$ the module
	\begin{equation*}
	\frac{\k\{\simplex \big( [m], [n] \big) \}}{\k\{\simplex_{\deg} \big( [m], [n] \big) \}},
	\end{equation*}
	and differential
	\begin{equation*}
	\partial(\tau) = \sum (-1)^i \tau \circ \delta_i.
	\end{equation*}
	To a morphism $\tau \colon [n] \to [n^\prime]$ it assigns the chain map
	\begin{equation*}
	\begin{tikzcd}[row sep=-3pt, column sep=normal,
	/tikz/column 1/.append style={anchor=base east},
	/tikz/column 2/.append style={anchor=base west}]
	\chains(\simplex^n)_m \arrow[r] &  \chains(\simplex^{n^\prime})_m \\
	\big( [m] \to [n] \big) \arrow[r, mapsto] & \big( [m] \to [n] \xra{\tau} [n^\prime] \big).
	\end{tikzcd}
	\end{equation*}
	We denote, as usual, the basis elements in $\chains(\simplex^n)_m$ by increasing tuples $[v_0, \dots, v_m]$ with $v_i \in \{0, \dots, n\}$.