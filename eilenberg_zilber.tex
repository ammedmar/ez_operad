\section{The Eilenberg--Zilber complex}

Let us consider a non-negative integer $r$.
The \textit{$r\th$ Eilenberg--Zilber complex} is the chain complex $\EZ(r)$ of natural transformations between the functors $\chains$ and $\chains^{\otimes r}$.
Explicitly, an element $f$ in $\EZ(r)$ is a collection of linear maps
\[
f = \left\{ f_n \in \Hom \big( \chains(\simplex^n), \ \chains(\simplex^n)^{\otimes r} \big) \right\}_{n \in \N}
\]
such that, for any simplicial map $\tau \colon [n] \to [n^\prime]$ the following diagram commutes:
\[
\begin{tikzcd}
\chains(\simplex^n) \arrow[r, "f_n"] \arrow[d, "\tau_\ast"] &
\chains(\simplex^n)^{\otimes r} \arrow[d, "\tau_\ast^{\otimes r}"] \\
\chains(\simplex^{n^\prime}) \arrow[r, "f_n"] &
\chains(\simplex^{n^\prime})^{\otimes r}.
\end{tikzcd}
\]
The grading and differential on $\EZ(r)$ are determined by its inclusion into the chain complex
\[
\prod_{n \in \N} \Hom \big( \chains(\simplex^n), \ \chains(\simplex^n)^{\otimes r} \big).
\]
By naturality, $f$ is determined by the collection
\[
\fhat = \left\{ \fhat_n \in \chains(\simplex^n)^{\otimes r}  \right\}_{n \in \N}
\]
where $\fhat_n = f_n(\id_{[n]})$.
Explicitly, for any basis element $\tau \colon [m] \to [n]$ in $\chains(\simplex^n)$ we have
\[
f_n(\tau) = f_n(\tau_\ast(\id_{[m]})) = \tau_\ast^{\otimes r}(f_m(\id_{[m]})) = \tau_\ast^{\otimes r}(\fhat_n).
\]
Notice that, also by naturality, this collection satisfies
\begin{equation} \label{e:normalized condition}
s_i^{\otimes r}(\fhat_n) = 0
\end{equation}
for all $i \in \{0, \dots, n\}$.
This bijection between $\EZ(r)$ and the submodule $\widehat{\EZ}(r)$ of
\[
\prod_{n \in \N} \chains(\simplex^n)^{\otimes r}
\]
satisfying condition \eqref{e:normalized condition} transfers a chain complex structure to $\widehat{\EZ}(r)$ with an element in $\fhat \in \widehat{\EZ}(r)$ having homogeneous of degree $m$ if $\deg(\fhat_n) = m+n$, and boundary recursively defined by
\begin{align*}
(\boundary \fhat)_0 &=
\boundary(\fhat_0), \\
(\boundary \fhat)_n &=
\boundary(\fhat_n) - (-1)^{m} \sum_{i=0}^{n} \delta_{i \ast}^{\otimes r}(\fhat_{n-1}).
\end{align*}